\begin{enumerate}
    \item[2.1\( ^*\)] Prove that if \( f: \mathbb{R}^n \rightarrow \mathbb{R}^m \) is differentiable at \( a \), then it is continuous at \( a \).
    
    \begin{proof}
    We shall present two proofs. Let \( \Delta f_a(h) = f(a+h)-f(a) \). For the first proof, let \( \epsilon > 0 \) be given. We observe that differentiability at \( a \) implies that
    \[
    \Delta f_a(h) = Df_a(h) + r(h)
    \]
    where \( \lim_{h\rightarrow0} \frac{\left| r(h) \right|}{\left| h \right|} = 0 \). Note that this then implies \( \lim_{h\rightarrow0} \left| r(h) \right| = 0 \). So then 
    \[
    \left| \Delta f_a(h) \right| =  \left| Df_a(h) + r(h) \right| \leq \left| Df_a(h) \right| + \left| r(h) \right|
    \]
    From linearity of \( Df_a \) and exercise 1.10, we know that there exists \( \lambda \) such that, for all \( h \), \( \left| Df_a(h) \right| \leq \lambda \left| h \right| \). Thus by choosing \( \delta \) small enough, \( \left| h \right| < \delta \) implies
    \[
    \left| \Delta f_a(h) \right| \leq \left| Df_a(h) \right| + \left| r(h) \right| < \lambda \left| h \right| + \left| r(h) \right| < \lambda \frac{\epsilon}{2\lambda} + \frac{\epsilon}{2} = \epsilon
    \]
    which implies that
    \[
    \lim_{h \rightarrow 0} f(a+h) = f(a)
    \]
    so that \( f \) is continuous at \( a \).
    
    For the second proof, we observe that given \( \epsilon > 0 \) we can choose \( \delta > 0 \) so that \( 0 < \left| h \right| < \delta \) implies
    \[
    \left| \left| \Delta f_a(h) \right| - \left| Df_a(h) \right| \right| < \left| \Delta f_a(h) - Df_a(h) \right| < \epsilon \left| h \right|
    \]
    Thus
    \[
    -\epsilon \left| h \right| < \left| \Delta f_a(h) \right| - \left| Df_a(h) \right| < \epsilon \left| h \right|
    \]
    so that
    \[
    \left| Df_a(h) \right| -\epsilon \left| h \right| < \left| \Delta f_a(h) \right| < \left| Df_a(h) \right| + \epsilon \left| h \right|
    \]
    Linearity of \( Df_a \) along with exercise 1.10 implies that, by taking the limit as \( h \rightarrow 0 \) and applying the squeeze theorem, we get that \( f \) is continuous at \( a \).
    \end{proof}

    \item[2.2] A function \( f: \mathbb{R}^2 \rightarrow \mathbb{R} \) is \emph{independent of the second variable} if for each \( x \in \mathbb{R} \) we have \( f(x,y_1) = f(x,y_2) \) for all \( y_1,y_2 \in \mathbb{R} \). Show that \( f \) is independent of the second variable if and only if there is a function \( g:\mathbb{R} \rightarrow \mathbb{R} \) such that \( f(x,y) = g(x) \). What is \( f'(a,b) \) in terms of \( g' \)?
    
    \begin{proof}
    If \( f \) is independent of the second variable, let \( x \in \mathbb{R} \) and set
    \[
    g(x) = f(x,x) = f(x,y)
    \]
    Thus \( f(x,y) = g(x) \). On the other hand, if \( g: \mathbb{R} \rightarrow \mathbb{R} \) is a function such that \( f(x,y) = g(x) \), then
    \[
    f(x,y_1) = g(x) = f(x,y_2) 
    \]
    implying that \( f \) is independent of the second variable. Now, assuming that \( f \) is differentiable at \( (a,b) \), we claim 
    \[
    f'(a,b) = \begin{bmatrix} g'(a) & 0 \end{bmatrix}
    \]
    To demonstrate, we see that there is a unique linear \( \lambda: \mathbb{R}^2 \rightarrow \mathbb{R} \) such that 
    \begin{align*}
        0 &= \lim_{(h,k)\rightarrow 0} \frac{\left| f(a+h,b+k)-f(a,b)-\lambda(h,k) \right|}{\left| (h,k) \right|} \\
        &= \lim_{h \rightarrow 0} \frac{\left| g(a+h)-g(a)-\lambda(h,k) \right|}{\left| h \right|}
    \end{align*}
    which implies then that \( g'(a) \) exists and
    \[
    \lambda(h,k) = g'(a) \cdot h = \begin{bmatrix} g'(a) & 0 \end{bmatrix} \begin{bmatrix} h \\ k \end{bmatrix}
    \]
    Thus
    \[
    f'(a,b) = \begin{bmatrix} g'(a) & 0 \end{bmatrix}
    \]
    \end{proof}
    
    \item[2.3] Define when a function \( f: \mathbb{R}^2 \rightarrow \mathbb{R} \) is independent of the first variable and find \( f'(a,b) \) for such \( f \). Which functions are independent of the first variable and also of the second variable?
    
    \begin{proof}
    A function \( f: \mathbb{R}^2 \rightarrow \mathbb{R} \) is independent of the first variable if and only if there is a \( g: \mathbb{R} \rightarrow \mathbb{R} \) such that \( g(y) = f(x,y) \). As before, it can be shown that, if \( f \) is differentiable at a point \( (a,b) \), then
    \[
    f'(a,b) = \begin{bmatrix} 0 & g'(b) \end{bmatrix}
    \]
    Functions which are independent of the first and the second variable are constant functions. 
    \end{proof}
    
    \item[2.4] Let \( g \) be a continuous real-valued function on the unit circle \( \{ x\in\mathbb{R}^2: \left| x \right| = 1 \} \) such that \( g(0,1) = g(1,0)=0 \) and \( g(-x) = -g(x) \). Define \( f: \mathbb{R}^2 \rightarrow \mathbb{R} \) by
    \[
    f(x) = \begin{cases} \left|x \right| g\left( \frac{x}{\left|x\right|} \right) & ,x \neq 0 \\ 0 & , x =0 \end{cases}
    \]
    
    \begin{enumerate}
        \item If \( x \in \mathbb{R}^2 \) and \( h: \mathbb{R} \rightarrow \mathbb{R} \) is defined by \( h(t) = f(tx) \), show that \( h \) is differentiable.
        
        \item Show that \( f \) is not differentiable at \( (0,0) \) unless \( g=0 \). 
    \end{enumerate}
    
    \begin{proof}
    \begin{enumerate}
        \item If \( x = 0 \in \mathbb{R}^2 \) then there is nothing to show. So, let \( x \neq 0 \), then 
        \begin{align*}
            \lim_{s \rightarrow 0} \frac{h(t+s) - h(t)}{s} &= \lim_{s \rightarrow 0} \frac{\left|(t+s)x\right|g \left( \frac{(t+s)x}{\left| (t+s)x \right|} \right)- \left| tx \right| g\left( \frac{tx}{\left|tx\right|} \right)}{s} \\
            \intertext{since \( t > 0 \) and \( s \) can always be chosen small enough, we get that}
            &= \left( \lim_{s \rightarrow 0} \frac{\left|t+s\right|g\left( \frac{(t+s)x}{\left| (t+s)x \right|} \right)-\left| t \right| g\left( \frac{tx}{\left|tx\right|} \right)}{s}\right) \left|x\right| \\
            &= \left| x \right| g\left( \frac{x}{\left| x \right|} \right)
        \end{align*}
    To justify the last equality, we observe that
    \[
        t > 0 \Rightarrow s \text{ can be chosen so that } t+s > 0
    \]
    which will produce the last equality.
    \[
        t < 0 \Rightarrow s \text{ can be chosen so that } t+s < 0 
    \]
    which will produce the last equality
    \[
        t = 0 \Rightarrow \text{regardless of } sgn(s) \text{ we get } \lim_{s \rightarrow 0} \frac{s g\left( \frac{x}{\left|x\right|} \right)}{s}
    \]
    which will produce the last equality.
    
    \item Suppose that \( Df(0,0) \) exists. Then there is \( \lambda \in \mathcal{L}(\mathbb{R}^2, \mathbb{R}) \) such that
    \begin{align*}
    0 &= \lim_{(h,k) \rightarrow (0,0)} \frac{\left| f(h,k) - f(0,0) - \lambda(h,k) \right|}{\left| (h,k) \right|} \\
    &= \lim_{(h,k) \rightarrow (0,0)} \frac{\left| f(h,k) - \lambda(h,k) \right|}{\left| (h,k) \right|}
    \end{align*}
    Approaching along the \(x\)-axis and \(y\)-axis separately, along with the linearity of \( \lambda \), implies that \( \lambda (x,y) = 0 \). That is, that \( Df(0,0) = 0 \). Thus
    \begin{align*}
        0 &= \lim_{(h,k) \rightarrow (0,0)} \frac{\left| f(h,k) \right|}{\left| (h,k) \right|} \\
        &= \lim_{(h,k) \rightarrow (0,0)} \left| g\left( \frac{(h,k)}{\left| (h,k) \right|} \right) \right|
    \end{align*}
    Now, let \( \epsilon > 0 \) be given and let \( (x,y) \in S^1 \). For \(  \left| \alpha \right| > 0 \) small enough, we will have that
    \[
    \left| g(x,y) \right| = \left| g\left( \frac{\alpha (x,y)}{\left| \alpha \right| \left| (x,y) \right|} \right) \right| < \epsilon
    \]
    So \( \left| g(x,y) \right| < \epsilon \) for every \( \epsilon > 0 \). Thus \( g(x,y) = 0 \). Therefore \( g = 0 \).
    \end{enumerate}
    \end{proof}
    
    \emph{Note: Continuity of \( g \) was not needed in 2.4} 

    \item[2.5] Let \( f: \mathbb{R}^2 \rightarrow \mathbb{R} \) be defined by
    \[
    f(x,y) = \begin{cases} \frac{x\left|y\right|}{\sqrt{x^2+y^2}} & , (x,y) \neq (0,0) \\ 0 & , (x,y) = (0,0) \end{cases}
    \]
    Show that \( f \) is a function of the kind considered in 2.4, so that \( f \) is not differentiable at \( (0,0) \).
    
    \begin{proof}
    Set \( g(x,y) = x \left| y \right| \). Note then that \( g \) is defined on \( S^1 \). It is simple to show that
    \[
    g(0,1) = g(1,0) = 0 \text{ and } g(-x) = -g(x) \text{ and } g \neq 0
    \]
    We also see that for \( (x,y) \neq (0,0) \) we have that
    \[
    \sqrt{x^2 + y^2}g\left( \frac{(x,y)}{\sqrt{x^2+y^2}} \right) = \sqrt{x^2 + y^2} \frac{x \left| y \right|}{x^2+y^2} = \frac{x\left|y\right|}{\sqrt{x^2+y^2}} = f(x,y)
    \]
    Thus, by 2.4, \( f \) is not differentiable at \( (0,0) \). 
    \end{proof}
    
    \item[2.6] Let \( f: \mathbb{R}^2 \rightarrow \mathbb{R} \) be defined by \( f(x,y) = \sqrt{\left|xy\right|} \). Show that \( f \) is not differentiable at \( (0,0) \). 
    
    \begin{proof}
    Notice, for \( (x,y) \neq (0,0) \)
    \[
    \left| (x,y) \right| f\left( \frac{(x,y)}{\left|(x,y)\right|} \right) = \sqrt{\left|xy\right|} = f(x,y)
    \]
    Thus, we have 
    \[
    f(x,y) = \begin{cases} \left| (x,y) \right| f\left( \frac{(x,y)}{\left|(x,y)\right|} \right) &, (x,y) \neq (0,0) \\ 0 &, (x,y) = (0,0) \end{cases}
    \]
    Furthermore, the conclusion in 2.4 (b) still holds when \( g(-x) = g(x) \). Thus, observing that
    \[
    f(0,1) = f(1,0) = 0 \text{ and } f(-(x,y)) = f(x,y) \text{ and } f \neq 0
    \]
    it follows from 2.4 (b) that \( f \) is not differentiable at \( (0,0) \). 
    \end{proof}
    
    \item[2.7] Let \( f: \mathbb{R}^n \rightarrow \mathbb{R} \) be a function such that \( \left| f(x) \right| \leq \left| x \right|^2 \). Show that \( f \) is differentiable at \( 0 \). 
    
    \begin{proof}
    We claim that \( Df(0,0) = 0 \). First, observe that
    \[
    0 \leq \left| f(0) \right| \leq \left| 0 \right|^2 = 0
    \]
    implying that \( f(0) = 0 \). Now, let \( \epsilon > 0 \) be given. Then \( 0 < \left| h \right| < \epsilon \) implies 
    \[
    \frac{\left| f(h) - f(0) - \lambda(h) \right|}{\left| h \right|} = \frac{\left| f(h) \right|}{\left| h \right|} \leq \left| h \right| < \epsilon
    \]
    \end{proof}
    
    \item[2.8] Let \( f: \mathbb{R} \rightarrow \mathbb{R}^2 \). Prove that \( f \) is differentiable at \( a \in \mathbb{R} \) if and only if \( f_1 \) and \( f_2 \) are, and that in this case
    \[
    f'(a) = \begin{bmatrix} f_1'(a) \\ f_2'(a) \end{bmatrix}
    \]
    
    \begin{proof}
    \begin{align*}
        0 &= \lim_{h \rightarrow 0} \frac{\left| f(a+h) - f(a) - \lambda(h) \right|}{\left|  h \right|} \\
        &= \lim_{h \rightarrow 0} \frac{\left| (f_1(a+h)-f_1(a)-\lambda_1(h), f_2(a+h)-f_2(a)-\lambda_2(h) )\right|}{\left| h \right|} \\
        &= \lim_{h \rightarrow 0} \left| \left( \frac{f_1(a+h)-f_1(a)-\lambda_1(h)}{h}, \frac{f_2(a+h)-f_2(a)-\lambda_2(h)}{h} \right) \right|
        \intertext{if and only if}
        (0,0) &= \lim_{h \rightarrow 0} \left( \frac{f_1(a+h)-f_1(a)-\lambda_1(h)}{h}, \frac{f_2(a+h)-f_2(a)-\lambda_2(h)}{h} \right)
    \end{align*}
    by Exercise 1.23, if and only if
    \begin{align*}
        0 &= \lim_{h \rightarrow 0} \frac{f_1(a+h)-f_1(a)-\lambda_1(h)}{h} \\
        \intertext{and}
        0 &= \lim_{h \rightarrow 0} \frac{f_2(a+h)-f_2(a)-\lambda_2(h)}{h}
    \end{align*}
    This establishes that \( Df(a) \) exists if and only if 
    \[
    f'(a) = \begin{bmatrix} (f_1)'(a) \\ (f_2)'(a) \end{bmatrix}
    \]
    \end{proof}
    
    \item[2.9] Two functions \( f,g: \mathbb{R} \rightarrow \mathbb{R} \) are equal up to \( n^{th} \) order at \( a \) if
    \[
    \lim_{h \rightarrow 0} \frac{f(a+h)-g(a+h)}{h^n} = 0
    \]
    \begin{enumerate}
        \item Show that \( f \) is differentiable at \( a \) if and only if there is a function \( g \) of the form \( g(x) = a_0 + a_1(x-a) \) such that \( f \) and \( g \) are equal up to the first order at \( a \)
        
        \item If \( f'(a), \ldots, f^{(n)}(a) \) exist, show that \( f \) and the function \( g \) defined by 
        \[
        g(x) = \sum_{i=0}^n \frac{f^{(i)}(a)}{i!}(x-a)^i
        \]
        are equal up to \( n^{th} \) order at \( a \).
    \end{enumerate}
    
    \begin{proof}
    \begin{enumerate}
        \item \( \Rightarrow \): If \( f \) is differentiable at \( a \) then simply set \( g(x) = f(a) + c (x-a) \) where \( f'(a) = c \).
        
        \vspace{1 cm}
        
        \( \Leftarrow \): We disagree, and offer a counter-example:
        \[
        f(x) = \begin{cases} x &, x \neq 0 \\ 1 &, x = 0  \end{cases}
        \]
        and \( g(x) = x \). Then
        \[
        \lim_{h \rightarrow 0} \frac{f(0+h) - g(0+h)}{h} = \lim_{h \rightarrow 0} \frac{h-h}{h} = 0
        \]
        Yet \( f \) is not continuous at \( 0 \), and so, a fortiori, is not differentiable at \( 0 \). However, if we insist that \( f \) be continuous at \( a \), then the proof follows from the fact that 
        \[
        \lim_{h \rightarrow 0} \frac{f(a+h) - g(a+h)}{h} = 0
        \]
        implies that
        \[
        \lim_{h \rightarrow 0} f(a+h) = a_0
        \]
        which, together with continuity, implies
        \[
        f(a) = a_0
        \]
        which gives the desired result.
        
        \item 
    \end{enumerate}
    
    \end{proof}
    
    \item[2.10] Use theorems of this section to find \( f' \) for the following:
    \begin{enumerate}
        \item \( f(x,y,z) = x^y \)
        
        \item \( f(x,y,z) = (x^y,z) \)
        
        \item \( f(x,y) = \sin(x \sin(y)) \)
        
        \item \( f(x,y,z) = \sin(x \sin (y \sin (z))) \)
        
        \item \( f(x,y,z) = x^{y^{z}} \)
        
        \item \( f(x,y,z) = x^{y+z} \)
        
        \item \( f(x,y,z) = (x+y)^z \)
        
        \item \( f(x,y) = \sin(xy) \)
        
        \item \( f(x,y) = \sin(xy)^{\cos(3)}\)
        
        \item \( f(x,y) = \left( \sin(xy), \sin(x \sin(y)), x^y \right) \)
    \end{enumerate}
    
    \begin{proof}
    \begin{enumerate}
        \item Notice \( x^y = e^{\ln(x^y)} \). Thus \( f(x,y,z) = exp\{\pi_2\ln(\pi_1)\}(x,y,z) \). So, if
        \begin{align*}
            g(x) &= e^x \\
            \intertext{and}
            h(x,y,z) &= \pi_2\ln(\pi_1) (x,y,z)
        \end{align*}
        then \( f(x,y,z) = (g \circ h) (x,y,z) \). It easily follows that
        \[
        g'(h(a,b,c)) = g'(b\cdot ln(a)) = \left[ a^b \right]
        \]
        On the other hand
        \begin{align*}
            Dh(a,b,c) &= \left[\ln(\pi_1) D\pi_2 + \pi_2 D\ln(\pi_1)   \right] (a,b,c) \\
            &= ln(a) D\pi_2(a,b,c) + b D \ln(\pi_1)(a,b,c) \\
            &= ln(a) \pi_2 + b D \ln(\pi_1)(a,b,c)
        \end{align*}
        and
        \begin{align*}
            D\ln(\pi_1)(a,b,c) &= D\ln(\pi_1(a,b,c)) \circ D\pi_1(a,b,c) \\
            &= D\ln(a) \circ \pi_1 \\
            &= \frac{1}{a} \circ \pi_1 \\
            &= \frac{1}{a}\pi_1
        \end{align*}
        Thus
        \[
        h'(a,b,c) = \left[ \begin{array}{ccc} 0 & \ln(a) & 0 \end{array} \right] + \left[ \begin{array}{ccc} \frac{b}{a} & 0 & 0 \end{array} \right] = \left[ \begin{array}{ccc} \frac{b}{a} & \ln(a) & 0 \end{array} \right]
        \]
        Therefore,
        \[
        f'(a,b,c) = g'(a,b,c) \cdot h'(a,b,c) = \left[ a^b \right] \cdot \left[ \begin{array}{ccc} \frac{b}{a} & \ln(a) & 0 \end{array}\right] = \left[ \begin{array}{ccc} ba^{b-1} & a^{b}\ln(a) & 0 \end{array} \right]
        \]
        
        \item From part (a) we know that 
        \[
        \left[x^y\right]'(a,b,c) = \left[ \begin{array}{ccc} ba^{b-1} & a^{b}\ln(a) & 0 \end{array} \right]
        \]
        By linearity, we also have
        \[
        \left[ z \right]' (a,b,c) = \left[ \begin{array}{ccc} 0 & 0 & 1  \end{array} \right]
        \]
        and so
        \[
        f'(a,b,c) = \left[ \begin{array}{ccc} ba^{b-1} & a^{b}\ln(a) & 0 \\ 0 & 0 & 1 \end{array}\right]
        \]
        
        \item Let \( f(x,y) = \sin( x \sin(y)) \) then, if 
        \[
        g(x) = \sin(x)
        \]
        and
        \[
        h(x,y) = x\sin(y) = \pi_1 \sin(\pi_2)
        \]
        then \( f(x,y) = (g \circ h) (x,y) \). So then 
        \[
        g'(h(a,b)) = \left[ \cos(a\sin(b)) \right]
        \]
        and
        \[
        Dh(a,b) = \sin(b)D\pi_1(a,b) + a D\sin(\pi_2)(a,b) = \sin(b)\pi_1 + a\cos(b)\pi_2
        \]
        so that
        \[
        f'(a,b) = \left[ \begin{array}{cc} \sin(b)\cos(a\sin(b)) & a\cos(b)\cos(a\sin(b)) \end{array} \right]
        \]
        
        \item Let \( f(x,y,z) = \sin(x \sin (y \sin (z))) \). If
        \[
        g(x) = \sin(x)
        \]
        and
        \[
        h(x,y,z) = x \sin (y \sin(z)) = \pi_1 \sin \circ (\pi_2 (\sin \circ \pi_3)) (x,y,z)
        \]
        Thus
        \[
        g_1' (g_2(a,b,c)) = \left[ \cos(a\sin(b\sin(c))) \right]
        \]
        and
        \[
        Dh(a,b,c) = \left[ \sin \circ (\pi_2 (\sin \circ \pi_3)) \right](a,b,c)D\pi_1(a,b,c) + \pi_1(a,b,c) D\sin \circ (\pi_2 (\sin \circ \pi_3))
        (a,b,c) \]
        and so by (c) and by an argument similar to those in exercises (2.2) and (2.3) we get
        \[
        g_2'(a,b,c) = \sin(b\sin(c)) \left[ \begin{array}{ccc} 1 & 0 & 0 \end{array} \right] + a \left[ \begin{array}{ccc} 0 & \sin(c)\cos(b\sin(c)) & b\cos(c)\cos(b\sin(c))  \end{array} \right]
        \]
        so that
        \[
        g_2' (a,b,c) = \left[ \begin{array}{ccc} \sin(b\sin(c)) & a\sin(c)\cos(b\sin(c)) & ab\cos(c)\cos(b\sin(c)) \end{array} \right]
        \]
        Therefore,
        \begin{align*}
            f'(a,b,c) &= g_1'(g_2(a,b,c))\cdot g_2'(a,b,c) \\
            &= \left[ \cos(a\sin(b\sin(c))) \right] \left[ \begin{array}{ccc} \sin(b\sin(c)) & a\sin(c)\cos(b\sin(c)) & ab\cos(c)\cos(b\sin(c)) \end{array} \right]
        \end{align*}
        
        \item Let \( f(x,y,z) = x^{y^z} \). Let
        \begin{align*}
            g(x,y,z) &= x^y \\
            \intertext{and}
            h(x,y,z) &= (x, y^z , 1) \\
            &= (\pi_1(x,y,z), \pi_2^{\pi_3}(x,y,z), 1)
        \end{align*}
        Thus 
        \[
        (g \circ h)(x,y,z) = g(h(x,y,z)) = g((x,y^z,0)) = x^{(y^z)^1} = x^{y^z} = f(x,y,z)
        \]
        Thus
        \[
        Df(a,b,c) = Dg(h(a,b,c)) \circ Dh(a,b,c)
        \]
        Recalling from part (a)
        \[
        g'(h(a,b,c)) = g'((x,y^z,1)) = \left[ \begin{array}{ccc} b^ca^{b^c-1} & a^{b^c}\ln(a) & 0 \end{array} \right]
        \]
        and
        \begin{align*}
            Dh(a,b,c) &= \left( D\pi_1(a,b,c), D\pi_2^{\pi_3}(a,b,c), D1(a,b,c) \right) \\
            &= \left( \pi_1, D\pi_2^{\pi_3}(a,b,c), 0 \right)
        \end{align*}
        Again, from part (a), we get that
        \[
        \left( \pi_2^{\pi_3} \right)'(a,b,c) = \left[ \begin{array}{ccc} 0 & cb^{c-1} & b^c\ln(b)  \end{array} \right]
        \]
        which gives us that
        \[
        h'(a,b,c) = \left[ \begin{array}{ccc} 1 & 0 & 0 \\ 0 & cb^{c-1} & b^c \ln(b) \\ 0 & 0 & 0 \end{array} \right]
        \]
        Therefore,
        \begin{align*}
            f'(a,b,c) &= g'(h(a,b,c)) \cdot h'(a,b,c) \\
            &= \left[ \begin{array}{ccc} b^ca^{b^c-1} & a^{b^c}\ln(a) & 0 \end{array} \right]  \left[ \begin{array}{ccc} 1 & 0 & 0 \\ 0 & cb^{c-1} & b^c\ln(b) \\ 0 & 0 & 0  \end{array} \right] \\
            &= \left[ \begin{array}{ccc} b^ca^{b^c-1} & cb^{c-1}a^{b^c}\ln(a) & a^{b^c}\ln(a)b^c\ln(b) \end{array}  \right]
        \end{align*}
    \end{enumerate}
    \end{proof}
    
    \item[2.11] Find \( f' \) for the following (where \( g: \mathbb{R} \rightarrow \mathbb{R} \) is continuous):
    \begin{enumerate}
        \item \( f(x,y) = \int_{a}^{x+y}g \)
        
        \item \( f(x,y) = \int_{a}^{x\cdot y}g \)
        
        \item \( f(x,y,z) = \int_{a}^{\sin(x \sin(y \sin(z)))}g \)
    \end{enumerate}
    
    \emph{Remark}: It's not too hard to show that if
    \[
    f(x) = \int_a^{h(x)} g
    \]
    with \( g \) continuous, then
    \[
    Df(c) = g(h(c)) \circ Dh(c)
    \]
    
    \begin{proof}
    \begin{enumerate}
        \item Let \( h(x,y) = x+y = \pi_1 + \pi_2 \). Then by our remark, we get
        \[
        Df(c,d) = g(c+d) \circ D(\pi_1+\pi_2)(c,d)
        \]
        which implies
        \[
        f'(c,d) = \left[ g(c+d) \right] \left[ \begin{array}{cc} 1 & 1 \end{array}\right] = \left[ \begin{array}{cc} g(c+d) & g(c+d) \end{array}\right]
        \]
        
        \item Let \( h(x,y) = xy = [\pi_1 \cdot \pi_2] (x,y) \). Then by our remark, we get
        \[
        Df(c,d) = g(cd) \circ D(\pi_1\pi_2)(c,d)
        \]
        which implies
        \[
        f'(c,d) = \left[ \begin{array}{cc} dg(cd) & cg(cd)  \end{array} \right]
        \]
        
        \item Let \( h(x,y,z) = \sin(x \sin(y \sin(z))) \). Then by our remark, we get
        \[
        Df(a,b,c) = g(\sin(a \sin(b \sin(c)))) \circ Dh(a,b,c)
        \]
        whose Jacobian is not worth writing out (see 2.10 (d)).
    \end{enumerate}
    \end{proof}
    
    \item[2.12] A function \( f: \mathbb{R}^n \times \mathbb{R}^m \rightarrow \mathbb{R}^p \) is bilinear if for \( x,x_1,x_2 \in \mathbb{R}^n \), \( y,y_1,y_2 \in \mathbb{R}^m \), and \( a \in \mathbb{R} \) we have
    \begin{center}
    \begin{tabular}{r l l l l}
        \( f(ax,y) \) & \( = \) & \( af(x,y) \) & \( = \) & \( f(x,ay) \) \\
        \( f(x_1+x_2,y) \) & \( = \) & \( f(x_1,y) \) & \( + \) & \( f(x_2,y) \) \\
        \( f(x,y_1+y_2) \) & \( = \) & \( f(x,y_1) \) & \( + \) & \( f(x,y_2) \)
    \end{tabular}
    \end{center}
    
    \begin{enumerate}
        \item Prove that if \( f \) is bilinear, then
        \[
        \lim_{(h,k) \rightarrow 0} \frac{\left| f(h,k) \right|}{\left| (h,k) \right|} = 0
        \]
        
        \item Prove that \( Df(a,b) (x,y) = f(a,y) + f(x,b) \)
        
        \item Show that the formula for \( Dp(a,b) \) in Theorem 2-3 is a special case of (b).
    \end{enumerate}
    
    \begin{lemma}
    If \( f: \mathbb{R}^n \times \mathbb{R}^m \rightarrow \mathbb{R}^p \) is bilinear, then there is \( M > 0 \) such that for all \( (a,b) \in \mathbb{R}^n \times \mathbb{R}^m \) with \( \left| (a,b) \right| \leq 1 \)
    \[
    \left| f(a,b) \right| \leq M 
    \]
    \end{lemma}
    \begin{proof}[Proof of Lemma]Let \( \{ e_1,\ldots,e_n \} \) and \( \{ s_1,\ldots,s_m \} \) be the standard bases of \( \mathbb{R}^n \) and \( \mathbb{R}^m \) respectively and let \( (a,b) \) be such that \( \left| (a,b) \right| \leq 1 \). Bilinearity of \( f \) yields 
    \[
    \left| f(a,b) \right| = \left| f\left( \sum_{i=1}^n \alpha_i e_i, \sum_{j=1}^m \beta_j s_j\right) \right| \leq \sum_{i=1}^n\sum_{j=1}^m \left| \alpha_i \right| \left| \beta_j \right| \left| f(e_i,s_j) \right|
    \]
    Setting \( M' = \max_{i,j}\{ \left| f(e_i,s_j) \right| \} \) gives us that
    \[
    \left| f(a,b) \right| \leq \sum_{i=1}^n\sum_{j=1}^m \left| \alpha_i \right| \left| \beta_j \right| M' 
    \]
    We observe that \( \left| a \right|, \left| b \right| \leq \left| (a,b) \right| \leq 1 \), which implies that \( \left| \alpha_i \right|, \left| \beta_j \right| \leq 1 \) for all \( i \) and \( j \). So we have
    \[
    \left| f(a,b) \right| \leq \sum_{i=1}^n\sum_{j=1}^m M' = n\cdot m \cdot M' = M
    \]
    \end{proof}
    \begin{proof}[Proof of Exercise]
    \begin{enumerate}
        \item Let \( \epsilon > 0 \) be given and \( 0 < \left| (h,k) \right| < \frac{\epsilon}{M} \) where \( M \) is as defined in the Lemma. Let \( \gamma = \left| (h,k) \right| \). Then
        \begin{align*}
            \frac{\left| f(h,k) \right|}{\gamma} &= \frac{\gamma^2 \left| f\left( \frac{1}{\gamma} (h,k) \right) \right|}{\gamma} \\
            &= \gamma \left| f\left( \frac{1}{\gamma} (h,k) \right) \right| \\
            &\leq \gamma M \\
            &< \epsilon
        \end{align*}
        
        \item Notice that
        \begin{align*}
            f(a+h,b+k) - f(a,b) - (f(a,k) + f(h,b)) &= f(a+h,b+k) - f(a,b) - f(a,k) - f(h,b)
        \end{align*}
        which, by bilinearity, yields
        \[
        = f(a,b) + f(a,k) + f(h,b) + f(h,k) - f(a,b) - f(a,k) - f(h,b) 
        \]
        \(\;\:\: = f(h,k) \)
        
        \vspace{1mm}
        
        which, by (a), will produce the result.
        
        \item It is a simple matter to show that \( p: \mathbb{R} \times \mathbb{R} \rightarrow \mathbb{R} \) defined by \( p(x,y) = x\cdot y \) is bilinear. Part (b) in the above implies then that 
        \[
        Dp(a,b) (x,y) = p(a,y) + p(x,b) = ay + bx = bx + ay
        \]
        as desired.
    \end{enumerate}
    \end{proof}
    
    \item[2.13] Define \( IP: \mathbb{R}^n \times \mathbb{R}^n \rightarrow \mathbb{R} \) by \( IP(x,y) = \langle x,y \rangle \)
    \begin{enumerate}
        \item Find \( D(IP)(a,b) \) and \( (IP)'(a,b) \)
        \item If \( f,g:\mathbb{R} \rightarrow \mathbb{R}^n \) are differentiable and \( h: \mathbb{R} \rightarrow \mathbb{R} \) is defined by \( h(t) = \langle f(t), g(t) \rangle \), show that 
        \[
        h'(a) = \langle f'(a)^T,g(a) \rangle + \langle f(a),g'(a)^T \rangle
        \]
        \item If \( f:\mathbb{R} \rightarrow \mathbb{R}^n\) is differentiable and \( \left| f(t) \right| = 1 \) for all \( t \), show that \( \langle f'(t)^T,f(t) \rangle = 0 \)
        \item Exhibit a differentiable function \( f:\mathbb{R} \rightarrow \mathbb{R} \) such that the function \( \left| f \right| \) defined by \( \left| f \right| (t)  = \left| f(t) \right| \) is not differentiable.
    \end{enumerate}
    \begin{proof}
    \begin{enumerate}
        \item It is a simple matter to show that \( IP \) is bilinear. So it follows from the previous exercise that
        \[
        DIP(a,b)(x,y) = IP(a,y) + IP(x,b) = \langle a,y \rangle + \langle x,b \rangle
        \]
        which implies
        \[
        (IP)'(a,b) = \left[ \begin{array}{cc} b & a \end{array} \right]
        \]
        where \( b \cdot x = IP (b,x) \) and \( a \cdot y = IP(a,y)\)
        
        \item Note that \( f(t) = (f_1(t),f_2(t),\ldots,f_n(t)) \) and \( (g_1(t),g_2(t),\ldots,g_n(t)) \). So
        \[
        h(t) = \langle f(t), g(t) \rangle = \sum_{i=1}^n f_i(t)g_i(t)
        \]
        which implies
        \[
        h'(a) = \sum_{i=1}^n (f_i\cdot g_i)'(a) = \sum_{i=1}^n \left[ g_i(a) \right] (f_i)'(a) + \left[ f_i(a) \right] (g_i)'(a)
        \]
        and so, considering \( [f'(a)]^T \) and \( [g'(a)]^T \) as elements of \( \mathbb{R}^n \), then we get
        \[
        h'(a) = \langle f'(a)^T, g(a) \rangle + \langle f(a), g'(a)^T \rangle
        \]
        
        \item Notice that \(  \left| f(t) \right| = 1 \) for all \( t \) implies that \( \langle f(t) , f(t) \rangle = 1 \) for all \( t \). Then if \( g(t) = \langle f(t), f(t) \rangle \), by Theorem 2-3 and (b) from this exercise, we get that for all \( a \) 
        \[
        0 = g'(a) = \langle f'(a)^T, f(a) \rangle + \langle f(a), f'(a)^T \rangle = 2\langle f'(a), f(a) \rangle 
        \]
        which implies that \( \langle f'(t)^T,f(t) \rangle = 0 \).
        
        \item \( f(x) = x^3 \) is such a function.
    \end{enumerate}
    \end{proof}


    \item Let \( E_i, i =1,2,\ldots,k \) be Euclidean spaces of various dimensions. A function \( f:E_1\times\ldots\times E_k \rightarrow \mathbb{R}^p \) is called multilinear if for each choice of \( x_j \in E_j \), \( j \neq i \) the function \( g:E_i \rightarrow \mathbb{R}^p \) defined by \( g(x) = f(x_1,\ldots, x_{i-1},x,x_{i+1},\ldots,x_k) \) is a linear transformation.
    \begin{enumerate}
        \item If \( f \) is multilinear and \( j \neq i \), show that for \( h = (h_1,h_2,\ldots,h_k) \), with \( h_l \in E_l \), we have
        \[
        \lim_{h \rightarrow 0} \frac{\left| f(a_1,\ldots,h_i,\ldots,h_j,\ldots,a_k) \right|}{\left| h \right|} = 0
        \]
        
        \item Prove that 
        \[
        Df(a_1,\ldots,a_k)(x_1,\ldots,x_k) = \sum_{i=1}^k f(a_1,\ldots,a_{i-1},x_i,a_{i+1},\ldots,a_k)
        \]
    \end{enumerate}
    \begin{proof}
    \begin{enumerate}
        \item It is a simple matter to show that if \( g(x,y) \) is defined by 
        \[
        g(x,y) = f(a_1,\ldots,x,\ldots,y,\ldots,a_k)
        \]
        then \( g \) is bilinear. Thus, by Exercise 2-12, we have
        \[
        \lim_{h \rightarrow 0} \frac{\left| f(a_1,\ldots,h_i,\ldots,h_j,\ldots,a_k) \right|}{\left| h \right|} = \lim_{h \rightarrow 0} \frac{\left| g(h_i,h_j) \right|}{\left| h \right|} = 0
        \]
        
        \item First, a simple inductive argument establishes that for \( n \geq 2 \) we will have
        \[
        \lim_{h \rightarrow 0} \frac{\left| f(h_1,\ldots,h_n) \right|}{\left| (h_1,\ldots, h_n) \right|} = 0
        \]
        To show this, we first observe that
        \begin{align*}
            f(h_1,\ldots,h_i,\ldots,h_n) &= f\left(h_1,\ldots,\sum_{k=1}^d\alpha_ke_k,\ldots,h_n  \right) \\
            &= \sum_{k=1}^d \alpha_kf(h_1,\ldots,e_i,\ldots,h_n)
        \end{align*}
        where \( \alpha_k(h_i) \rightarrow 0 \) as \( h_i \rightarrow 0 \). So, if the claim holds for \( n \) then, by the inductive hypothesis along with our previous observation, we have
        \[
            \lim_{h \rightarrow 0} \frac{\left| f(h_1,\ldots,h_i,\ldots,h_{n+1}) \right|}{\left| (h_1,\ldots,h_i,\ldots,h_{n+1}) \right|} \leq \lim_{h \rightarrow 0} \sum_{k=1}^d \left| \alpha_k \right|\frac{\left| f(h_1,\ldots,e_k,\ldots,h_{n+1}) \right|}{\left| (h_1,\ldots,e_k,\ldots,h_{n+1}) \right|} = 0
        \]
        Now, it's not too hard to show that the difference
        \begin{align*}
        &= f(a_1+h_1,\ldots,a_k+h_k) - f(a_1,\ldots,a_k)-\sum_{i=1}^k f(a_1,\ldots,h_i,\ldots,a_k)\\
        &= f(a_1,\ldots,a_k) + \sum_{i=1}^k f(a_1,\ldots,h_i,\ldots,a_k) + \sigma - f(a_1,\ldots,a_k) \\
        & \hspace{5mm}- \sum_{i=1}^k f(a_1,\ldots,h_i,\ldots,a_k) \\
        &= \sigma(h_1,\ldots,h_k)
        \end{align*}
        where 
        \begin{multline*}
        \sigma = \sum f(a_1,\ldots,h_{i_1},\ldots,h_{i_2},\ldots,a_k) + \\ \sum f(a_1,\ldots,h_{i_1},\ldots,h_{i_2},\ldots,h_{i_3}\ldots, a_k) + \ldots + f(h_1,\ldots,h_k)
        \end{multline*}
        To be clear, \( \sigma \) is simply the sum of all images by \( f \) where there are two or more \( h\)'s in the argument. Thus,
        \begin{align*}
            &= \lim_{h \rightarrow 0} \frac{\left|f(a_1+h_1,\ldots,a_k+h_k) - f(a_1,\ldots,a_k)-\sum_{i=1}^k f(a_1,\ldots,h_i,\ldots,a_k) \right|}{\left| h \right|}\\
            &= \lim_{h \rightarrow 0} \frac{\left| \sigma \right|}{\left| h \right|} \\
            \intertext{and by our previous considerations, we get}
            &= 0
        \end{align*}
    \end{enumerate}
    \end{proof}
    
    \item[2.15] Regard an \( n \times n \) matrix as a point in the \(n\)-fold product \( \mathbb{R}^n \times \ldots \times \mathbb{R}^n \) by considering each row as a member of \( \mathbb{R}^n \).
    \begin{enumerate}
        \item Prove that \( det: \mathbb{R}^n \times \ldots \times \mathbb{R}^n \rightarrow \mathbb{R}\) is differentiable and
        \[
        D(det)(a_1,\ldots,a_n)(x_1,\ldots,x_n) = \sum_{i=1}^n det \begin{bmatrix} a_1 \\ . \\ . \\ . \\ x_i \\ . \\ . \\ . \\ a_n \end{bmatrix}
        \]
        
        \item If \( a_{ij}: \mathbb{R} \rightarrow \mathbb{R} \) are differentiable and \( f(t) = det(a_{ij}(t)) \), show that
        \[
        f'(t) = \sum_{j=1}^n det \begin{bmatrix} a_{11}(t) & \ldots & a_{1n}(t) \\ \vdots & & \vdots \\ a_{j1}'(t) & \ldots & a_{jn}'(t) \\ \vdots & & \vdots \\ a_{n1}(t) & \ldots & a_{nn}(t) \end{bmatrix}
        \]
        
        \item If \( det(a_{ij}(t)) \neq 0 \) for all \( t \) and \( b_1,\ldots, b_n:\mathbb{R} \rightarrow \mathbb{R} \) are differentiable, let \( s_1,\ldots,s_n: \mathbb{R} \rightarrow \mathbb{R} \) be the functions such that \( s_{1}(t),\ldots,s_n(t) \) are the solutions of the equations
        \[
        \sum_{j=1}^n a_{ji}(t)s_{j}(t) = b_i(t) \;\;\;\; i = 1,2,\ldots, n
        \]
        Show that \( s_i \) is differentiable and find \( s_i'(t) \).
    \end{enumerate}
    
    \begin{proof}
    \begin{enumerate}
        \item It is well-known that the determinant, \( det \), is multilinear in both the rows and the columns. Thus, the result follows immediately from the previous exercise part (b).
        
        \item Let \( g: \mathbb{R} \rightarrow \mathbb{R}^n \times \ldots \times \mathbb{R}^n \) be defined by
        \[
        g(t) = \begin{bmatrix} a_{11}(t) & \ldots & a_{1n}(t) \\ \vdots & & \vdots \\ a_{n1}(t) & \ldots & a_{nn}(t) \end{bmatrix} = \begin{bmatrix} g_1(t) \\ \vdots \\ g_n(t) \end{bmatrix}
        \]
        Then
        \[
        f = det \circ g
        \]
        and thus
        \[
        Df(k) = Ddet(g(k)) \circ Dg(k) 
        \]
        Now, we know
        \[
        Ddet(g(k))(x_1,\ldots,x_n) = \sum_{i=1}^n det \begin{bmatrix} g_1(k) \\ \vdots \\ x_i \\ \vdots \\ g_n(k) \end{bmatrix}
        \]
        which, by Theorem 2-3, implies that
        \[
        Ddet(g(k)) \circ D(g(k)) = \sum_{i=1}^n det \begin{bmatrix} g_1(k) \\ \vdots \\ Dg_i(k) \\ \vdots \\ g_n(k) \end{bmatrix}
        \]
        which, by \( Ddet(c) = det'(c) \) and, again, Theorem 2-3, implies that
        \[
        f'(t) = \sum_{i=1}^n det \begin{bmatrix} g_1(k) \\ \vdots \\ g'_i (k) \\ \vdots \\ g_n(k) \end{bmatrix} = \sum_{i=1}^n det \begin{bmatrix} a_{11}{k} & \ldots & a_{1n}(k) \\ \vdots & & \vdots \\ a'_{i1}(k) & \ldots & a'_{in}(k) \\ \vdots & & \vdots \\ a_{n1}(k) & \ldots & a_{nn}(k) \end{bmatrix}
        \]
        
        \item The first part follows from Cramer's rule. To find \( s'(t) \) it is enough to apply (b) and the quotient rule. 
    \end{enumerate}
    \end{proof}
    
    \item[2.16] Suppose \( f :\mathbb{R}^n \rightarrow \mathbb{R}^n \) is differentiable and has a differentiable inverse \( f^{-1}: \mathbb{R}^n \rightarrow \mathbb{R}^n \). Show that \( (f^{-1})'(a) = \left[ f'(f^{-1}(a)) \right]^{-1} \).

    \begin{proof}
    \[
        f \circ f^{-1} (x) = I(x) = x
    \]
    implies
    \[
    I_{n} = ( f \circ f^{-1})'(a) = f'(f^{-1}(a))\cdot (f^{-1})'(a)
    \]
    which implies
    \[
    (f^{-1})'(a) = \left[ f'(f^{-1}(a)) \right]^{-1}
    \]
    \end{proof}

    \item[2.17] Find the partial derivatives of the following functions:
    
    \item[2.18] Find the partial derivatives of the following functions (where \( g: \mathbb{R} \rightarrow \mathbb{R} \) is continuous):
    \begin{enumerate}
        \item \( f(x,y) = \int_{a}^{x+y}g \)
        
        \item \( f(x,y) = \int_y^x g \)
        
        \item \( f(x,y) = \int_a^{\int_b^y g} g\)
    \end{enumerate}

    \begin{proof}
    \begin{enumerate}
        \item For \( 1 \leq i \leq 2 \)
        \begin{align*}
            D_if(c,d) &= \lim_{h \rightarrow 0} \frac{\int_a^{c+d+h}g - \int_a^{c+d}g}{h} \\
            &= \lim_{h \rightarrow 0} \frac{G(c+d+h)-G(c+d)}{h}\\
            &= g(c+d)
        \end{align*}
        
        \item 
        \begin{align*}
            D_1f(c,d) &= D_1[G(x)-G(y)](c,d) \\
            &= g(c)\\
            D_2f(c,d) &= D_2[G(x)-G(y)](c,d) \\
            &= -g(d)
        \end{align*}
        
        \item 
        \begin{align*}
            D_1f(c,d) &= 0 \\
            D_2f(c,d) &= g(G(d)-G(b))\cdot g(d)
        \end{align*}
    \end{enumerate}
    \end{proof}
    
    \item[2.19] If \( f(x,y) = x^{x^{x^{x^{y}}}}+\log(x)(\arctan(\arctan(\arctan(\sin(\cos xy)-\log(x+y))))) \) find \( D_2f(1,y) \). 
    
    \begin{proof}
    \( f(1,y) = 1 \) implying \( D_2f(1,y) = 0 \). 
    \end{proof}

    \item[2.20] Find the partial derivatives of \( f \) in terms of the derivatives of \( g \) and \( h \) if
    \begin{enumerate}
        \item \( f(x,y) = g(x)h(y) \)
        
        \item \( f(x,y) = g(x)^{h(y)} \)
        
        \item \( f(x,y) = g(x) \)
        
        \item \( f(x,y) = g(y) \)
        
        \item \( f(x,y) = g(x+y) \)
    \end{enumerate}
    
    \begin{proof}
    \begin{enumerate}
        \item \( D_{1}f(a,b) = h(b)g'(a)\) and \( D_{2}f(a,b) = g(a)h'(b) \).
        
        \item \( D_{1}f(a,b) = h(b)g(a)^{h(b)}g'(a) \) and \( D_2f(a,b) = g(a)^{h(b)}\ln(g(a))h'(b) \)
        
        \item \( D_{1}f(a,b) = g'(a) \) and \( D_2f(a,b) = 0 \)
        
        \item \( D_{1}f(a,b) = 0 \) and \( D_2f(a,b) = h'(b) \)
        
        \item \( D_{1}f(a,b) = g'(a+b) = D_{2}f(a,b) \)
    \end{enumerate}
    \end{proof}
    
    \item[2.21] Let \( g_1,g_2:\mathbb{R}^2 \rightarrow \mathbb{R} \) be continuous. Define \( f: \mathbb{R}^2 \rightarrow \mathbb{R} \) by
    \[
    f(x,y) = \int_0^x g_1(t,0)dt + \int_0^yg_2(x,t)dt
    \]
    \begin{enumerate}
        \item Show that \( D_2f(x,y) = g_2(x,y) \).
        \item How should \( f \) be defined so that \( D_1f(x,y) = g_1(x,y) \)?
        \item Find a function \( f:\mathbb{R}^2 \rightarrow \mathbb{R} \) such that \( D_1f(x,y) = x \) and \( D_2f(x,y) = y \). Find one such that \( D_1f(x,y) = y \) and \( D_2f(x,y) = x \).
    \end{enumerate}
    \begin{proof}
    \begin{enumerate}
        \item Let us define
        \begin{align*}
            f_1(x) &= \int_0^x g_1(t,0)dt \\
            f_{2}(y) &= \int_0^yg_2(x,t)dt
        \end{align*}
        Then \( f(x,y) = f_1(x) + f_2(y) \) which implies that
        \[
        D_2f(x,y) = f_{2}'(y) = g_2(x,y)
        \]
        
        \item If \( f(x,y) = \int_0^x g_1(t,y)dt + \int_0^yg_2(0,t)dt \) then \( D_2f(x,y) = g_1(x,y) \).
        
        \item Let \( f(x,y) = \frac{x^2}{2} + \frac{y^2}{2} \). Then
        \[
        D_1 f(x,y) = x \text{ and } D_{2}f(x,y) = y
        \]
        If \( f(x,y) = xy \) then 
        \[
        D_1f(x,y) = y \text{ and } D_2f(x,y) = x
        \]
    \end{enumerate}
    \end{proof}
    
    \item[2.22\(^*\)] If \( f: \mathbb{R}^2 \rightarrow \mathbb{R} \) and \( D_2f = 0 \), show that \( f \) is independent of the second variable. If \( D_1f=D_2f=0 \), show that \( f \) is constant.
    
    \begin{proof}
    Let \( f: \mathbb{R}^2 \rightarrow \mathbb{R} \) such that \( D_2f = 0 \). Given \( x \), if we set \( g(y) = f(x,y) \) then \( g'(y) = D_2f(x,y) = 0 \) for all \( y \), implying, by MVT, that \( g(y) \) is constant. But then 
    \[
    f(x,y_1) = g(y_1) = g(y_2) = f(x,y_2)
    \]
    so that \( f \) is independent of the second variable. Now, if \( D_1f = D_2f = 0 \) then \(D_1f = 0 \) implies that \( f \) is independent of the first variable and \( D_2f = 0 \) implies that \( f \) is independent of the second variable. So
    \[
    f(x_1,y_1) = f(x_2,y_2)
    \]
    implying that \( f \) is constant.
    \end{proof}

    \item[2.23\(^*\)] Let \( A = \{ (x,y) \in \mathbb{R}^2: x<0, \text{ or } x \geq 0 \text{ and } y \neq 0 \} \)
    \begin{enumerate}
        \item If \( f:A \rightarrow \mathbb{R} \) and \( D_1f = D_2f = 0 \), show that \( f \) is constant. 
        
        \item Find a function \( f:A \rightarrow \mathbb{R} \) such that \( D_2f = 0 \) but \( f \) is not independent of the second variable.
    \end{enumerate}

    \begin{proof}
    \begin{enumerate}
        \item Let \( (x_1,y_1), (x_2,y_2) \in A \). Notice there exists a sequence of lines, each parallel to one of the axes, going from \( (x_1,y_1) \) to \( (x_2,y_2) \) so that no line passes through the origin. So, if \( (x^{(i)},y^{(i)}) \) is the appropriate endpoint of the \( i^{th} \) line, then from Exercise 2.22, we have
        \[
        f(x_1,y_2) = f(x',y') = f(x'',y'') = \ldots =  f(x_2,y_2)
        \]
        so that \( f \) is constant.
        
        \item Notice that
        \[
        f(x) = \begin{cases} 1 &, (0,y) \text{ where } y > 0 \\ 0 &, \text{ elsewhere }  \end{cases}
        \]
        is such a function.
    \end{enumerate}
    \end{proof}

    \item[2.24] Define \( f: \mathbb{R}^2 \rightarrow \mathbb{R} \) by
    \[
    f(x,y) = \begin{cases} xy \frac{x^2-y^2}{x^2+y^2} &, (x,y) \neq 0 \\ 0 &, (x,y) = 0 \end{cases}
    \]
    \begin{enumerate}
        \item Show that \( D_2f(x,0) = x \) for all \( x \) and \( D_1f(0,y) = -y \) for all \( y \)
        \item Show that \( D_{1,2}f(0,0) \neq D_{2,1}f(0,0)\).
    \end{enumerate}
    
    \begin{proof}
    \begin{enumerate}
        \item It is a simple matter to show that
        \[
        D_2f(x,0) = \lim_{h \rightarrow 0} \frac{f(x,h)-f(x,0)}{h} = x
        \]
        and
        \[
        D_1f(0,y) = \lim_{h \rightarrow 0} \frac{f(h,y)-f(0,y)}{h} = -y
        \]
        
        \item It is a simple matter to show that
        \[
        D_{1,2}f(0,0) = -1 \neq 1 = D_{2,1}f(0,0)
        \]
    \end{enumerate}
    \end{proof}

    \item[2.25\(^*\)] Define \( f:\mathbb{R} \rightarrow \mathbb{R} \) by
    \[
    f(x) = \begin{cases} e^{-x^{-2}} &, x \neq 0 \\ 0 &, x=0 \end{cases}
    \]
    Show that \( f \) is a \( C^\infty \) function, and \( f^{(i)}(0) = 0 \) for all \( i \).
    \begin{proof}
    \end{proof}
    
    \item[2.26\(^*\)] Too bored. Will return later.
    \begin{proof}
    
    \end{proof}
    \item[2.27] Too bored. Will return later.
    \begin{proof}
    
    \end{proof}
    
    \item[2.28] Find expressions for the partial derivatives of the following functions:
    \begin{enumerate}
        \item \( F(x,y) = f(g(x)k(y), g(x)+h(y)) \)
        
        \item \( F(x,y,z) = f(g(x+y), h(y+z)) \)
        
        \item \( F(x,y,z) = f(x^y, y^z, z^x) \)
        
        \item \( F(x,y) = f(x,g(x),h(x,y)) \)
    \end{enumerate}
    
    \begin{proof}
    \begin{enumerate}
        \item Let us define
        \begin{align*}
            q_1(x,y) &= g(x)k(y) \\
            q_2(x,y) &= g(x) + h(y)
        \end{align*}
        Then
        \begin{align*}
            q_1'(x,y) &= \begin{bmatrix} k(y)g'(x) & g(x)k'(y) \end{bmatrix} \\
            q_2'(x,y) &= \begin{bmatrix} g'(x) & h'(y) \end{bmatrix}
        \end{align*}
        and 
        \[
        F(x,y) = f(q_1(x,y), q_2(x,y))
        \]
        By Theorem 2.9, we get that
        \begin{align*}
            D_1F(a,b) &= D_1f(q_1(a,b), q_2(a,b))k(b)g'(a) + D_2f(q_1(a,b), q_2(a,b))g'(a) \\
            D_2F(a,b) &= D_1f(q_1(a,b), q_2(a,b))g(a)k'(b) + D_2f(q_1(a,b), q_2(a,b))h'(b)
        \end{align*}
        
        \item We see that
        \begin{align*}
            (g \circ (\pi_1+\pi_2))'(x,y,z) &= \begin{bmatrix} g'(x+y) & g'(x+y) & 0 \end{bmatrix} \\
            \intertext{and}
            (h \circ (\pi_2+\pi_3))'(x,y,z) &= \begin{bmatrix} 0 & h'(y+z) & h'(y+z) \end{bmatrix}
        \end{align*}
        and so if \( r = (g(a+b), h(b+c) ) \), then
        \begin{align*}
            D_1F(a,b,c) &= D_1f(r)g'(a+b) \\
            D_2F(a,b,c) &= D_1f(r)g'(a+b)+D_2f(r)h'(b+c) \\
            D_3F(a,b,c) &= D_2f(r)h'(b+c)
        \end{align*}
        
        \item Skipping the details, if \( r = (a^b, b^c, c^a) \) then
        \begin{align*}
            D_1F(a,b,c) &= D_1f(r) ba^{b-1}+D_3f(r)\ln(c)c^a \\
            D_2F(a,b,c) &= D_1f(r) \ln(a)a^b + D_2f(r)cb^{c-1} \\
            D_3F(a,b,c) &= D_2f(r)\ln(b)b^c+D_3f(r)ac^{a-1}
        \end{align*}
        
        \item Skipping the details, if \( r = (a, g(a), h(a,b)) \) then
        \begin{align*}
            D_1F(a,b) &= D_1f(r) + D_2f(r) g'(a) + D_3f(r)D_1h(a,b) \\
            D_2F(a,b) &= D_3f(r) D_2h(a,b)
        \end{align*}
    \end{enumerate}
    \end{proof}
    
    \item[2.29] Let \( f: \mathbb{R}^n \rightarrow \mathbb{R} \). For \( x \in \mathbb{R}^n \), the limit
    \[
    \lim_{t \rightarrow 0} \frac{f(a+tx)-f(a)}{t}
    \]
    if it exists, is denoted \( D_xf(a) \), and called the \textbf{directional derivative} of \( f \) at \( a \), in the direction of \( x \).
    \begin{enumerate}
        \item Show that \( D_{e_i}f(a) = D_if(a) \).
        \item Show that \( D_{tx}f(a) = tD_xf(a) \).
        \item If \( f \) is differentiable at \( a \), show that \( D_xf(a) = Df(a)(x) \) therefore \( D_{x+y}f(a) = D_xf(a)+D_yf(a) \).
    \end{enumerate}
    
    \begin{proof}
    \begin{enumerate}
        \item Notice that \( a+he_i = (a_1,\ldots,a_i+h,\ldots,a_n) \). Specifically,
        \begin{align*}
            D_{e_i}f(a) &= \lim_{h \rightarrow 0} \frac{f(a+he_i)-f(a)}{h} \\
            &= \lim_{h \rightarrow 0} \frac{f(a_1,\ldots,a_i+h,\ldots,a_n)-f(a_1,\ldots,a_n)}{h} \\
            &= D_if(a)
        \end{align*}
        
        \item 
        \begin{align*}
            D_{\alpha x}f(a) &= \lim_{t \rightarrow 0} \frac{f(a+t\alpha x)-f(a)}{t} \\
            &= \lim_{t \rightarrow 0} \frac{f(a+\frac{t}{\alpha}\alpha x) - f(a)}{\frac{t}{\alpha}} \\
            &= \alpha \lim_{t \rightarrow 0} \frac{f(a+tx)-f(a)}{t} \\
            &= \alpha D_xf(a)
        \end{align*}
        
        \item Let \( f \) be differentiable at \( a \). If \( g(t) = a+tx \) then 
        \begin{align*}
        D_xf(a) &= \lim_{t \rightarrow 0} \frac{f(a+tx)-f(a)}{t} \\
        &= \lim_{t \rightarrow 0} \frac{(f\circ g)(t)-(f \circ g)(0)}{t} \\
        &= D(f\circ g)(0) \\
        &= Df(g(0)) \circ Dg(0) \\
        &= Df(a)(x)
        \end{align*}
        By linearity of \( Df(a) \) we get the desired result.
    \end{enumerate}
    \end{proof}

    \item[2.30] Let \( f \) be defined as in Problem 2.4. Show that \( D_xf(0,0) \) exists for all \( x \), but if \( g \neq 0 \), then \( D_{x+y}f(0,0) = D_xf(0,0) + D_yf(0,0) \) is not true for all \( x \) and \( y \). 
    
    \begin{proof}
    We see from 2.4(a) that \( D_xf(0,0) = h'(0) = \left| x \right| g\left( \frac{x}{\left|x \right|} \right)\) for \( x \neq 0 \) and \( D_{(0,0)}f(0,0) = 0 \). If \( g \neq 0 \), then there exists \( z \neq 0 \) such that \linebreak \( g\left( \frac{z}{\left| z \right|} \right) \neq 0 \). If \( x = (\pi_1(z),0) \) and \( y = (0,\pi_2(z)) \), then
    \[
    D_{x+y}f(0,0) = D_{z}f(0,0) = \left| z \right| g\left( \frac{z}{\left| z \right|} \right) \neq 0
    \]
    while, by our assumptions on \( g \), we have
    \[
    D_xf(0,0) + D_yf(0,0) = 0 + 0 = 0
    \]
    which shows that 
    \[
    D_{x+y}f(0,0) \neq D_xf(0,0) + D_yf(0,0)
    \]
    \end{proof}
    
    \item[2.31] Let \( f: \mathbb{R}^2 \rightarrow \mathbb{R} \) be defined as in Problem 1.26. Show that \( D_xf(0,0) \) exists for all \( x \), although \( f \) is not even continuous at \( (0,0) \). 
    
    \begin{proof}
    Recall that from 1.26(a), we have that every straight line through \( (0,0) \) contains an interval about \( (0,0) \) which is contained in \( \mathbb{R}^2\setminus A \), where \( A \) is as defined in 1.26. This means \( D_xf(0,0) \) exists for all \( x \) and
    \[
    D_xf(0,0) = 0
    \]
    \end{proof}
    
    \item[2.32]
    \begin{enumerate}
        \item Let \( f: \mathbb{R} \rightarrow \mathbb{R} \) be defined by
        \[
        f(x) = \begin{cases} x^2\sin\left( \frac{1}{x} \right) &, x \neq 0 \\ 0 &, x = 0 \end{cases}
        \]
        Show that \( f \) is differentiable at \( 0 \) but that \( f' \) is not continuous at \( 0 \).
    
        \item Let \( f: \mathbb{R}^2 \rightarrow \mathbb{R} \) be defined by
        \[
        f(x,y) = \begin{cases} (x^2+y^2) \sin \left( \frac{1}{\sqrt{x^2+y^2}} \right) &, (x,y) \neq 0  \\ 0 &, (x,y) = 0 \end{cases}
        \]
        Show that \( f \) is differentiable at \( (0,0) \) but that \( D_if \) is not continuous at \( (0,0) \). 
    \end{enumerate}
    \begin{proof}
    \begin{enumerate}
        \item We see that
        \[
        f'(0) = 0
        \]
        and that, for \( x \neq 0 \)
        \[
        f'(x) = 2x\sin(1/x)-\cos(1/x)
        \]
        which is clearly not continuous at \( 0 \).
        
        \item We see that
        \[
        \lim_{h \rightarrow 0} \frac{f(0+h)-f(0)}{\left| h \right|} = \lim_{h \rightarrow 0} \frac{\left|h\right|^2 \sin\left( 1/\left|h\right| \right)}{\left| h \right|} = \lim_{h \rightarrow 0} \left| h \right| \sin(1/\left|h \right|) = 0
        \]
        so that \( Df(0,0) = 0 \). With a little calculation, we can see that for \( (x,y) \neq (0,0) \) we get
        \[
        D_if(x,y) = 2 \cdot \pi_i(x,y) \cdot \sin\left( \frac{1}{\sqrt{x^2+y^2}} \right)-\pi_i(x,y) \cdot \frac{\cos\left( \frac{1}{\sqrt{x^2+y^2}} \right)}{\sqrt{x^2+y^2}}
        \]
        It is not too hard to show that when \( (x,y) \rightarrow (0,0) \) the first term tends to zero but that the term on the end is pathological (Check \( \lim_{x \rightarrow 0^+} D_1f(x,0) \) for example). Therefore, \( D_if \) is not continuous at \( (0,0) \). 

    \end{enumerate}
    \end{proof}
    
    \item[2.33] Show that continuity of \( D_1f_i \) at \( a \) may be eliminated from the hypothesis of Theorem 2-8.
    
    \begin{proof}
    This is because existence of \( D_1f_j(a) \) will imply that
    \begin{multline*}
    \lim_{h\rightarrow 0} \frac{\left| f_j(a+h) - f_j(a) - \sum_{i=1} D_if_j(a)(h_i) \right|}{\left| h \right|} = \\ \lim_{h \rightarrow 0} \frac{\left| f_j(a_1+h_1,a_2\ldots,a_n)-f_j(a_1,\ldots,a_n)-D_1f_j(a)(h_1) \right|}{\left| h \right|} + \frac{\left| \sum_{i=2}^n [D_if_j(c_i)-D_if_j(a)] h_i \right|}{\left| h \right|}
    \end{multline*}
    so that continuity of only the other partials is need to obtain the desired limit.
    \end{proof}
    
    \item[2.34] A function \( f: \mathbb{R}^n \rightarrow \mathbb{R} \) is \textbf{homogeneous} of degree \( m \) if \( f(tx)=t^mf(x) \) for all \( x \). If \( f \) is also differentiable, show that
    \[
    \sum_{i=1}^n x_iD_if(x) = mf(x)
    \]
    
    \begin{proof}
    If we let \( g(t) = f(tx) \) and \( h(t) = tx \), then 
    \[
    g(t) = (f \circ h)(t)
    \]
    From this it is not too difficult to show that
    \[
    g'(1) = \sum_{i=1}^n x_iD_if(x)
    \]
    Now, if \( f \) is homogeneous of degree \( m \), then
    \[
    g(t) = f(tx) = t^mf(x)
    \]
    so that
    \[
    g'(t) = m f(x) t^{m-1}
    \]
    and
    \[
    g'(1) = mf(x)
    \]
    \end{proof}
    
    \item[2.35] If \( f: \mathbb{R}^n \rightarrow \mathbb{R} \) is differentiable and \( f(0) = 0 \), prove that there exist \( g_i:\mathbb{R}^n \rightarrow \mathbb{R} \) such that
    \[
    f(x) = \sum_{i=1}^n x_ig_i(x)
    \]
    
    \begin{proof}
    The author's hint is very revealing. Let \( h_x(t) = f(tx) \). So then
    \[
    \int_0^1 h_x'(t)dt = h_x(1) - h_x(0) = f(x) - f(0) = f(x)
    \]
    However, we can see that
    \[
    h_x'(t) = \sum_{i=1}^n x_iD_if(tx) 
    \]
    which implies that
    \begin{align*}
        f(x) &= \int_0^1 h_x'(t)dt \\
        &= \sum_{i=1}^n x_i \int_0^1 D_if(tx)dt
    \end{align*}
    which implies that if \( g_i(x) = \int_0^t D_if(tx)dt \), we obtain the desired result.
    \end{proof}

    \item[2.36] Let \( A \subset \mathbb{R}^n \) be an open set and \( f: A \rightarrow \mathbb{R}^n \) a continuously differentiable 1-1 function such that \( det f'(x) \neq 0 \) for all \( x \). Show that \( f(a) \) is an open set and \( f^{-1}:f(A) \rightarrow A \) is differentiable. Show also that \( f(B) \) is open for any open set \( B \subset A \).
    
    \begin{proof}
    Note that the assumptions on \( f \) and \( A \) imply that the Inverse Function Theorem holds on \( f \) for all \( x \in A \). So, given \( y \in f(A) \) there are open \( W_y \subset f(A) \) and \( V_x \subset A \) and a function \( \Tilde{f}^{-1}: W_y \rightarrow V_x \) which is continuous and differentiable. But by injectivity of \( f \), we know that \( f^{-1}: f(A) \rightarrow A \) exists and \( f^{-1}(z) = \Tilde{f}^{-1}(z) \) for \( z \in W_y \). So \( f^{-1} \) is continuous and differentiable at \( y \). Therefore, \( f^{-1} \) is differentiable and continuous on \( f(A) \). Continuity of \( f^{-1} \) implies both that \( f(A) \) is open, and that \( f(B) \) is open whenever \( B \subset A \) is (Willard).
    \end{proof}
    
    \item[2.37]
    \begin{enumerate}
        \item Let \( f: \mathbb{R}^2 \rightarrow \mathbb{R} \) be a continuously differentiable function. Show that \( f \) is \emph{not} 1-1.
        
        \item Generalize this result to the case of a continuously differentiable function \( f: \mathbb{R}^n \rightarrow \mathbb{R}^m \) with \( m<n \).
    \end{enumerate}
    \begin{proof}
    \begin{enumerate}
        \item Assuming $f$ is continuously differentiable (though we could use a weaker assumption than this) and injective, it follows that $f(x,y)$ is differentiable on $\mathbb{R}^2$ and therefore continuous on $\mathbb{R}^2$. So $f(x,0)$ and $f(0,y)$ are continuous and injective on the interval $[-1,1]\subset \mathbb{R}$. By connectedness of $[-1,1]$ and continuity of $f(x,0)$ and $f(0,y)$, the images of $[-1,1]$ by $f(x,0)$ and $f(0,y)$ are both connected. $f(0,0)$ being a maximum or minimum of either function on $[-1,1]$ contradicts with $f$ being injective. If $f(0,0)$ is not a maximum or a minimum, then it follows by connectedness that both images contain some open interval about $f(0,0)$, which, again, contradicts with injectivity of $f$. 
        
        \item Let \( f:\mathbb{R}^n \rightarrow \mathbb{R}^m \), \( m < n \), be continuously differentiable. Suppose for contradiction that \( f \) is injective. If we define \( g: \mathbb{R}^n \rightarrow \mathbb{R}^n \) by 
        \[
        g(x_1,\ldots,x_n) = \left( x_1,x_2,\ldots,x_{n-m},f(x_1,\ldots,x_n) \right)
        \]
        then we know that
        \[
        g'(x) = \left[ \begin{array}{c|c} Id & 0 \\ \hline * & \frac{\partial f}{\partial \vec{y}}(x) \end{array} \right]
        \]
        where
        \[
       \frac{\partial f}{\partial \vec{y}}(x) = \begin{bmatrix} D_{n-m+1}f_1(x) & \ldots & D_{n}f_1(x) \\ \vdots & & \vdots \\ D_{n-m+1}f_m(x) & \ldots & D_nf_m(x) \end{bmatrix}
        \]
        and \( * \) is unimportant since  \( det g'(x) \neq 0 \) iff \( det \frac{\partial f}{\partial \vec{y}}(x) \neq 0 \). Now, if \( det \frac{\partial f}{\partial \vec{y}}(x) = 0 \) then there exists \( v \in \mathbb{R}^m \) such that \( v \neq 0 \) and
        \[
        \left[ \frac{\partial f}{\partial \vec{y}}(x)\right] v = 0
        \]
        which implies that if \( v^* \in \mathbb{R}^n \) admitting \( v^* = (0,\ldots,0,v) \) then we get that
        \[
        D_{v^*}f(x) = Df(x)(v^*) = \left[ \frac{\partial f}{\partial \vec{y}}(x)\right] v = \vec{0} \in \mathbb{R}^m
        \]
    \end{enumerate}
    \end{proof}
    
    \item[2.38]\begin{enumerate}
        \item If \( f: \mathbb{R} \rightarrow \mathbb{R} \) satisfies \( f'(a) \neq 0 \) for all \( a \in \mathbb{R} \), show that \( f \) is not injective (on all \( \mathbb{R} \)).
        
        \item Define \( f:\mathbb{R}^2 \rightarrow \mathbb{R}^2 \) by \( f(x,y) = (e^x \cos(y), e^x \sin(y)) \). Show that \( det f'(x,y) \neq 0 \) for all \( (x,y) \in \mathbb{R}^2 \) but \( f \) is not injective.
    \end{enumerate}
    
    \begin{proof}
    \begin{enumerate}
        \item This is a simple consequence of the Mean Value Theorem. 
        
        \item We see that
        \[
        f'(x,y) = \begin{bmatrix} \cos(y)e^x & -e^x\sin(y) \\ \sin(y)e^x & e^x\cos(y) \end{bmatrix}
        \]
        so that
        \[
        det f'(x,y) = e^{2x}(\cos^2(y)+\sin^2(y)) = e^{2x} \neq 0 \:\:\: \forall (x,y)
        \]
        and yet \( f(0,0) = (1,0) = f(0, 2\pi) \).
    \end{enumerate}
    \end{proof}
    
    \item[2.39] Use the function \( f: \mathbb{R} \rightarrow \mathbb{R} \) defined by
    \[
    f(x) = \begin{cases} \frac{x}{2} + x^2 \sin\left( \frac{1}{x} \right) & x \neq 0 \\ 0 & x = 0 \end{cases}
    \]
    To show that continuity of the derivative cannot be eliminated from the hypotheses in theorem 2-11.
    
    \begin{proof}
    Observe that
    \[
    f'(0) = \lim_{h \rightarrow 0} \frac{f(h)}{h} = \lim_{h \rightarrow 0}\frac{1}{2} + h \sin\left(\frac{1}{h}\right) = \frac{1}{2}
    \]
    and, for \( x \neq 0 \)
    \[
    f'(x) = \frac{1}{2} + 2x\sin\left(\frac{1}{x} \right) -\cos\left( \frac{1}{x} \right)
    \]
    so that \( f'(0) \neq 0 \) but \( f'(x) \) is continuously differentiable only when \( x \neq 0 \). One could argue by relatively elementary means that
    \[
    f\left( \frac{1}{(2n+1)\pi} \right)  < f\left( \frac{1}{(n+1)2\pi} \right) < f\left( \frac{1}{n2\pi} \right) 
    \]
     which implies that \( f \) is not injective on any interval about \( 0 \). 
    \end{proof}
    
    \item[2.40] Use the implicit function theorem to re-do Problem 2-15(c).
    \begin{proof}
    The assumptions in 2-15(c) are not strong enough to enable us to use the Implicit Function Theorem. If we assume that each \( a_{ij}(t) \) and \( b_i(t) \) are continuously differentiable, then letting \( [A(t)]_{ij} = a_{ij}(t) \), \( s= (s_1,\ldots,s_n) \in \mathbb{R}^n \) and \( b: \mathbb{R} \rightarrow \mathbb{R}^n \) with \( b(t) = (b_1(t),\ldots,b_n(t)) \), we can define \( f: \mathbb{R}\times \mathbb{R}^n \rightarrow \mathbb{R}^n \) by
    \[
    f(t,s) = A(t)^Ts-b(t)
    \]
    By non-singularity of \( A(t) \), we know that for every \( t \) there is \( s \) such that \( f(t,s) = 0 \). Furthermore, by \( a_{ij}(t) \) and \( b_{i}(t) \) being continuously differentiable, we know that \( D_1f(t,s) \) exists and is continuous. It is easy to show that for all \( i \geq 2 \), \( D_{i}f(t,s) \) exists and is continuous since \( f(t,s) \) is affine in \( s \). Indeed, if \( M \) is as defined in the theorem, then \( M = A \) and so is non-singular for all \( t \). Thus, by the Implicit Function Theorem, we will have that for all \( t \), we can find \( s(t) \) so that \( f(t,s(t)) = 0 \) with \( s \) differentiable at \( t \). 
    \end{proof}
    
    \item[2.41] Let \( f: \mathbb{R} \times \mathbb{R} \rightarrow \mathbb{R} \) be differentiable. For each \( x \in \mathbb{R} \) define \( g_x: \mathbb{R} \rightarrow \mathbb{R} \) by \( g_x(y) = f(x,y) \). Suppose that for each \( x \) there is a unique \( y \) with \( g_x'(y) = 0 \); let \( c(x) \) be this \( y \).
    \begin{enumerate}
        \item If \( D_{2,2} f(x,y) \neq 0 \) for all \( (x,y) \), show that \( c \) is differentiable and
        \[
        c'(x) = -\frac{D_{2,1}f(x,c(x))}{D_{2,2}f(x,c(x))}
        \]
        
        \item Show that if \( c'(x) = 0 \), then for some \( y \) we have 
        \begin{align*}
            D_{2,1}f(x,y) &= 0 \\
            D_2f(x,y) &= 0
        \end{align*}
        
        \item Let \( f(x,y) = x(y \log y - y)-y \log(x) \). Find
        \[
        \max_{\frac{1}{2} \leq x \leq 2}\left( \min_{\frac{1}{3} \leq y \leq 1} f(x,y) \right)
        \]
    \end{enumerate}
    
    \begin{proof}
    \begin{enumerate}
        \item If we just assume that \( D_2f(x,y) \) is continuously differentiable, then we will get that, for \( M \) as defined in Theorem 2-12, \( M = D_{2,2}f(x,c(x)) \) so that \( det \: M \neq 0 \) for all \( x \in \mathbb{R} \). The Implicit Function Theorem then implies that \( c'(x) \) exists for all \( x \in \mathbb{R} \). To compute \( c'(x) \) we simply see that if we let \( h(x) = (x, c(x)) \), then \( [D_2f] \circ h = 0 \). This implies
        \begin{align*}
        0 &= \left( D_2 f \circ h \right)'(x) \\
        &= \left[ D_{2,1}f(x,c(x)) \;\; D_{2,2} f(x,c(x)) \right] \begin{bmatrix} 1 \\ c'(x) \end{bmatrix} \\
        &= D_{2,1}f(x,c(x)) + D_{2,2}f(x,c(x))c'(x)
        \end{align*}
        which implies the desired result.
        
        \item Note that \( g_x'(y) = D_2f(x,y) \). So, if \( c'(x) = 0 \), then, letting \( y = c(x) \) yields
        \[
        g_x'(y) = D_2f(x,y) = 0 = -\frac{D_{2,1}f(x,y)}{D_{2,2}f(x,y)}
        \]
        
        \item Maybe some other time.
    \end{enumerate}
    \end{proof}

\end{enumerate}
